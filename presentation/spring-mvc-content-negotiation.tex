\section{Négociation de contenu}
\begin{frame}
 \frametitle{Outline}
 \scriptsize{\tableofcontents[currentsection,currentsubsection]}
\end{frame}


\begin{frame}
 \frametitle{Représentation}
 \begin{itemize}
  \item En REST, aucun format n'est imposé
  \begin{itemize}
   \item En SOAP, XML est imposé
  \end{itemize}
  \item Le client et le serveur négocie le format
  \item Tout se passe avec deux entêtes
  \begin{itemize}
   \item \code{Accept} : dans la requête, ce que le client comprend
   \item \code{Content-Type} : dans la réponse, ce que le serveur renvoie
  \end{itemize}
 \end{itemize}
\end{frame}

\begin{frame}[fragile]
 \frametitle{\code{Accept} et \code{Content-Type}}
 \begin{itemize}
  \item Requête
 \end{itemize}
 \begin{textcode}
GET /content-negotiation/zen-contact/contacts/1 HTTP/1.1
Accept: application/xml, text/xml, application/*+xml, application/json
Host: localhost:8080
 \end{textcode}

 \begin{itemize}
  \item Réponse
 \end{itemize}
 \begin{textcode}
HTTP/1.1 200 OK
Content-Type: application/xml
<?xml version="1.0" encoding="UTF-8" standalone="yes"?>
<contact>
  <age>37</age>
  <firstname>Joe</firstname>
  <id>1</id>
  <lastname>Dalton</lastname>
</contact>
 \end{textcode}

\end{frame}

\begin{frame}
 \frametitle{Dans Spring MVC}
 \begin{itemize}
  \item Spring MVC gère de façon transverse
  \begin{itemize}
   \item La négociation de contenu
   \item La désérialisation/sérialisation de la requête/réponse
  \end{itemize}
  \item Conséquences pour le dévelopeur :
  \begin{itemize}
   \item Un même contrôleur peut retourner plusieurs types de contenu
   \item Travail sur des objets de domaine
  \end{itemize}
 \end{itemize}

\end{frame}

\begin{frame}[fragile]
 \frametitle{Désérialisation/sérialisation}
 
 \begin{javacode}
// d\'es\'erialiser le corps de la requ\^ete pour obtenir un Contact
@RequestMapping(value="/contacts",method=RequestMethod.POST)
@ResponseStatus(HttpStatus.CREATED)
public void create(@RequestBody Contact contact, 
                   HttpServletRequest request, 
                   HttpServletResponse response) {
  (...)
}

// s\'erialiser le Contact retourn\'e
// puis le mettre dans le corps de la r\'eponse
@RequestMapping(value="/contacts/{id}",method=RequestMethod.GET)
@ResponseBody
public Contact contact(@PathVariable Long id) {    
  return contactRepository.findOne(id);
}
 \end{javacode}

\end{frame}

\begin{frame}[fragile]
 \frametitle{Désérialisation/sérialisation}
 
 \begin{itemize}
  \item Idem avec le \code{RestTemplate}
 \end{itemize}

 
 \begin{javacode}
// d\'es\'erialiser le corps de la r\'eponse pour obtenir un Contact
Contact contact = tpl.getForObject(
  "http://localhost:8080/crud-rest/zen-contact/contacts/1", 
  Contact.class,
  id
);

// s\'erialiser le Contact \`a cr\'eer
// puis le mettre dans le corps de la requ\^ete
Contact contact = new Contact();
URI location = tpl.postForLocation(
  "http://localhost:8080/crud-rest/zen-contact/contacts",
  contact
);
 \end{javacode}

\end{frame}

\begin{frame}[fragile]
 \frametitle{Désérialisation/sérialisation, qui ?}
 
 \begin{itemize}
  \item Des \code{HttpMessageConverter}s
  \item Autour du contrôleur (dans le \code{HandlerAdapter})
  \item Dans le \code{RestTemplate}
 \end{itemize}

\end{frame}

\begin{frame}[fragile]
 \frametitle{\code{HttpMessageConverter}}
 
 \begin{javacode}
public interface HttpMessageConverter<T> {

  boolean canRead(Class<?> clazz, MediaType mediaType);

  boolean canWrite(Class<?> clazz, MediaType mediaType);

  List<MediaType> getSupportedMediaTypes();

  // d\'es\'erialisation (repr\'esentation vers objet)
  T read(Class<? extends T> clazz, HttpInputMessage inputMessage)
      throws IOException, HttpMessageNotReadableException;
  
  // s\'erialisation (objet vers repr\'esentation)
  void write(T t, MediaType contentType, 
             HttpOutputMessage outputMessage)
      throws IOException, HttpMessageNotWritableException;

}
 \end{javacode}

\end{frame}

\begin{frame}[fragile]
 \frametitle{\code{HttpMessageConverter} dans Spring MVC}
 
 \begin{itemize}
  \item Spring MVC connait
  \begin{itemize}
   \item Le format (\code{Accept} ou \code{Content-Type})
   \item Le type Java attendu (en paramètre ou en retour)
  \end{itemize}
  \item Spring MVC consulte ses \code{HttpMessageConverter}s
  \item Il utilise celui qui peut faire la conversion
 \end{itemize}

\end{frame} 
 
\begin{frame}
 \frametitle{\code{HttpMessageConverter}s disponibles}
 
 \begin{itemize}
  \item JAXB2 (XML), Jackson (JSON), Atom, RSS, Spring OXM (XML)
  \item Formulaire HTML, \code{byte[]}, etc.
  \item Automatiquement enregistrés (si librairie tierce présente)
 \end{itemize}

\end{frame} 

\begin{frame}[fragile]
 \frametitle{Déclarer un \code{HttpMessageConverter}s coté serveur} 
 
 \begin{xmlcode}
<mvc:annotation-driven>
  <mvc:message-converters register-defaults="true">
    <bean class="com.zenika.nordnet.web.CsvHttpMessageConverter" />
  </mvc:message-converters>
</mvc:annotation-driven>
 \end{xmlcode}

\end{frame} 

\begin{frame}[fragile]
 \frametitle{Déclarer un \code{HttpMessageConverter}s coté client} 
 
 \begin{javacode}
RestTemplate tpl = new RestTemplate();
List<HttpMessageConverter<?>> convs = 
  new ArrayList<HttpMessageConverter<?>>();
convs.add(new MappingJacksonHttpMessageConverter());
tpl.setMessageConverters(convs);
 \end{javacode}

\end{frame} 

\begin{frame}
 \frametitle{Intercepteurs} 
 
 \begin{itemize}
  \item Spring MVC propose des intercepteurs, cotés client et serveur
  \item Pratique pour des traitements transverses ou systématiques
 \end{itemize}

\end{frame} 

\begin{frame}[fragile]
 \frametitle{Intercepteur coté serveur} 
 
 \begin{javacode}
public interface HandlerInterceptor {
  
  boolean preHandle(HttpServletRequest request, 
                    HttpServletResponse response, 
                    Object handler) throws Exception;

  void postHandle(HttpServletRequest request, 
                  HttpServletResponse response, 
                  Object handler, ModelAndView modelAndView)
             throws Exception;
  
  void afterCompletion(HttpServletRequest request, 
                       HttpServletResponse response, 
                       Object handler, Exception ex)
             throws Exception;

}
 \end{javacode}

\end{frame} 

\begin{frame}[fragile]
 \frametitle{Déclarer un intercepteur coté serveur} 
 
 \begin{xmlcode}
<mvc:interceptors>
  <bean class="com.zenika.nordnet.web.LogHandlerInterceptor" />
</mvc:interceptors>
 \end{xmlcode}

\end{frame} 

\begin{frame}[fragile]
 \frametitle{Intercepteur coté client} 
 
 \begin{javacode}
public interface ClientHttpRequestInterceptor {

  ClientHttpResponse intercept(HttpRequest request, byte[] body, 
                               ClientHttpRequestExecution execution)
                       throws IOException;

}
 \end{javacode}

\end{frame} 

\begin{frame}[fragile]
 \frametitle{Déclarer un intercepteur coté client} 
 
 \begin{javacode}
RestTemplate tpl = new RestTemplate();
List<ClientHttpRequestInterceptor> interceptors = 
  new ArrayList<ClientHttpRequestInterceptor>();
interceptors.add(new LogClientHttpRequestInterceptor());
tpl.setInterceptors(interceptors);
 \end{javacode}

\end{frame} 