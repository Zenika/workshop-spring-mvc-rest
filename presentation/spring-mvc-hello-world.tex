\section{Spring MVC}
\input{current-section.tex}

\begin{frame}
 \frametitle{Spring MVC en deux mots}
 \begin{itemize}
  \item Framework pour applications web ``classiques''
  \item Support pour web services REST
   \begin{itemize}
    \item Support serveur et client (\code{RestTemplate})
   \end{itemize}
  \item Dépendances : Spring et API Servlet
  \item Très flexible (beaucoup de points d'extension)
  \item Base technique pour d'autres frameworks
  \begin{itemize}
   \item Spring Web Flow, Grails
  \end{itemize}
 \end{itemize}
\end{frame}

\begin{frame}
 \frametitle{\code{DispatcherServlet}}
 
 \begin{itemize}
  \item Le coeur de Spring MVC
  \item Coordonne des composants d'infrastructure
  \item Démarre son propre contexte Spring, il contient :
  \begin{itemize}
   \item Les contrôleurs de l'application
   \item Des beans d'infrastructure
  \end{itemize}
 \end{itemize}
\end{frame}

\begin{frame}[fragile]
 \frametitle{Déclarer une \code{DispatcherServlet} dans le \code{web.xml}}
 
 \begin{xmlcode}
<servlet>
  <servlet-name>spring-mvc</servlet-name>
  <servlet-class>
    org.springframework.web.servlet.DispatcherServlet
  </servlet-class>
  <init-param>
    <param-name>contextConfigLocation</param-name>
    <param-value>/WEB-INF/spring-mvc-servlet.xml</param-value>
  </init-param>
  <load-on-startup>1</load-on-startup>
</servlet>

<servlet-mapping>
  <servlet-name>spring-mvc</servlet-name>
  <url-pattern>/zen-contact/*</url-pattern>
</servlet-mapping>
 \end{xmlcode}
 
\end{frame}

\begin{frame}[fragile]
 \frametitle{Fichier Spring de la \code{DispatcherServlet}}
 
  \begin{itemize}
   \item Si pas de fichier précisé, utilise un chemin par défaut
   \item Convention : \code{/WEB-INF/\$\{servlet-name\}-servlet.xml}
  \end{itemize}
  
 \begin{xmlcode}
<!-- utilise my-app-servlet.xml -->
<servlet>
  <servlet-name>my-app</servlet-name>
  <servlet-class>
    org.springframework.web.servlet.DispatcherServlet
  </servlet-class>  
</servlet>
 \end{xmlcode}
 
\end{frame}

\begin{frame}[fragile]
 \frametitle{Contrôleur ``Hello World''}
 
  \begin{itemize}
   \item Les contrôleurs sont des POJO annotés
  \end{itemize}
  
 \begin{javacode}
@Controller // indique \`a Spring que c'est un contr\^oleur
public class HelloWorldController {

  @RequestMapping("/hello") // quel URL ?
  @ResponseBody // mettre l'objet retourn\'e dans le corps de la r\'eponse
  public String hello() {
    return "Hello World!";
  }
  
}
 \end{javacode}
 
\end{frame}

\begin{frame}[fragile]
 \frametitle{Les contrôleurs sont des beans Spring}
 
  \begin{itemize}
   \item Il faut bien déclarer les contrôleurs dans Spring
   \item Une solution est le component scanning
   \begin{itemize}
    \item Déclarations en XML ou en Java fonctionnent aussi
   \end{itemize}
  \end{itemize}
  
 \begin{xmlcode}
<context:component-scan base-package="com.zenika.web" />
 \end{xmlcode}
 
\end{frame}

\begin{frame}
 \frametitle{Root application context}
 
  \begin{itemize}
   \item Contexte Spring de l'application web
   \item Indépendant de Spring MVC
   \item Contient généralement
   \begin{itemize}
    \item Services, repositories
    \item \code{DataSource}, transaction manager
   \end{itemize}
  \end{itemize}
 
\end{frame}

\begin{frame}[fragile]
 \frametitle{Déclarer le root application context dans le \code{web.xml}}
  
 \begin{xmlcode}
<context-param>
  <param-name>contextConfigLocation</param-name>
  <param-value>
    /WEB-INF/applicationContext.xml
  </param-value>
</context-param>

<listener>
  <listener-class>
    org.springframework.web.context.ContextLoaderListener
  </listener-class>
</listener>
 \end{xmlcode}
 
\end{frame}

\begin{frame}
 \frametitle{Root application context et \code{DispatcherServlet}}
 
 \begin{itemize}
  \item La \code{DispatcherServlet} se branche sur le root application context
  \begin{itemize}
   \item S'il a été déclaré
  \end{itemize}
  \item Relation parent-enfant entre les contextes Spring
  \item Beans du root contexte visibles pour les beans de la \code{DispatcherServlet}
  \begin{itemize}
   \item Pas l'inverse
  \end{itemize}
 \end{itemize}
 
\end{frame}

\begin{frame}[fragile]
 \frametitle{Root application context et \code{DispatcherServlet}}
 
 \begin{xmlcode}
<!-- Dans le contexte de la DispatcherServlet -->
<bean class="com.zenika.web.HelloWorldController">
  <!-- contactService est d\'eclar\'e dans le root context -->
  <property name="contactService" 
            ref="contactService />
</bean>
 \end{xmlcode}

\end{frame}
 
 \begin{frame}
 \frametitle{Configuration Java, sans \code{web.xml}}
 
 \begin{itemize}
  \item Possibilité de configurer Spring en Java
  \item Depuis Servlet 3.0, configuration sans \code{web.xml}
  \item 0 XML !
 \end{itemize}
 
\end{frame}

\begin{frame}[fragile]
 \frametitle{Configuration de Spring MVC en Java}
 
 \begin{javacode}
@Configuration // c'est une classe de configuration
@ComponentScan(basePackageClasses=HelloWorldController.class)
@EnableWebMvc  // configuration REST de Spring MVC
public class HelloWorldConfiguration 
       extends WebMvcConfigurerAdapter { // surcharger des m\'ethodes 
                                         // (si n\'ecessaire)
	
}
 \end{javacode}

\end{frame}

\begin{frame}[fragile]
 \frametitle{Initialisation de l'application}
 
 \begin{javacode}
public class HelloWorldWebApplicationInitializer
       implements WebApplicationInitializer {

  @Override
  public void onStartup(ServletContext servletContext)
              throws ServletException {
    // cr\'eation root application context
    // et servlet
  }

}
 \end{javacode}
 
\end{frame}

\begin{frame}[fragile]
 \frametitle{Initialisation de l'application (root context)}
 
 \begin{javacode}
public class HelloWorldWebApplicationInitializer
       implements WebApplicationInitializer {

  @Override
  public void onStartup(ServletContext servletContext)
              throws ServletException {

    AnnotationConfigWebApplicationContext rootContext =
      new AnnotationConfigWebApplicationContext();
    rootContext.register(BackendApplicationConfiguration.class);
    servletContext.addListener(new ContextLoaderListener(rootContext));
    (...)
  }

}
 \end{javacode}
 
\end{frame}

\begin{frame}[fragile]
 \frametitle{Initialisation de l'application (dispatcher servlet)}
 
 \begin{javacode}
public class HelloWorldWebApplicationInitializer
       implements WebApplicationInitializer {

  @Override
  public void onStartup(ServletContext servletContext)
              throws ServletException {
    (...)
    AnnotationConfigWebApplicationContext dispatcherContext =
      new AnnotationConfigWebApplicationContext();
    dispatcherContext.register(HelloWorldConfiguration.class);
    ServletRegistration.Dynamic dispatcher = servletContext.addServlet(
      "dispatcher", new DispatcherServlet(dispatcherContext)
    );
    dispatcher.setLoadOnStartup(1);
    dispatcher.addMapping("/zen-contact/*");
  }

}
 \end{javacode}
 
\end{frame}
