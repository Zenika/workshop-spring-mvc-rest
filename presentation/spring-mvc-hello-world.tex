\section{Spring MVC}
\input{current-section.tex}

\begin{frame}
 \frametitle{Spring MVC en deux mots}
 \begin{itemize}
  \item Framework pour applications web ``classiques''
  \item Support pour web services REST
   \begin{itemize}
    \item Support serveur et client (\code{RestTemplate})
   \end{itemize}
  \item Dépendances : Spring et API Servlet
  \item Très flexible (beaucoup de points d'extension)
  \item Base technique pour d'autres frameworks
  \begin{itemize}
   \item Spring Web Flow, Grails
  \end{itemize}
 \end{itemize}
\end{frame}

\begin{frame}
 \frametitle{\code{DispatcherServlet}}

 \begin{itemize}
  \item Le coeur de Spring MVC
  \item Coordonne des composants d'infrastructure
  \item Appelle les contrôleurs applicatifs
 \end{itemize}
\end{frame}


\begin{frame}[fragile]
 \frametitle{Contrôleur ``Hello World''}

  \begin{itemize}
   \item Les contrôleurs sont des POJO annotés
  \end{itemize}

 \begin{javacode}
@RestController // indique \`a Spring que c'est un contr\^oleur
public class HelloWorldController {

  @RequestMapping("/hello") // quel URL ?
  public String hello() {
    return "Hello World!";
  }

}
 \end{javacode}

\end{frame}

\begin{frame}
 \frametitle{Les contrôleurs sont des beans Spring}

  \begin{itemize}
   \item Il faut bien déclarer les contrôleurs dans Spring
   \item Une solution est le component scanning
   \begin{itemize}
    \item Déclarations en XML ou en Java fonctionnent aussi
   \end{itemize}
  \end{itemize}

\end{frame}

\begin{frame}[fragile]
 \frametitle{Comment démarrer ?}

  \begin{itemize}
    \item Utiliser Spring Boot
    \item Gère les dépendances, la \code{DispatcherServlet}, etc.
  \end{itemize}

  \begin{javacode}
 @SpringBootApplication // g\`ere notamment le component scanning
 public class SpringMvcOverviewApplication {

   public static void main(String[] args) {
     SpringApplication.run(SpringMvcOverviewApplication.class, args);
   }

 }
  \end{javacode}

\end{frame}
