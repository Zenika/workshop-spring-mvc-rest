\section{Principes de REST}
\input{current-section.tex}

\begin{frame}
 \frametitle{REST}
 \begin{itemize}
  \item Representational State Transfer
  \item Un style d'architecture
  \item Une façon de faire communiquer des applications
  \item Utiliser HTTP comme un protocole applicatif
  \begin{itemize}
   \item Pas juste comme un protocole de transport
  \end{itemize}
 \end{itemize}
\end{frame}

\begin{frame}
 \frametitle{Principes}
 
 \begin{itemize}
  \item Ressources identifiées
  \item Interface uniforme
  \item Sans état
  \item Représentation
  \item Hypermedia
 \end{itemize} 

\end{frame}

\begin{frame}
 \frametitle{Ressources identifiées}
 
 \begin{itemize}
  \item Tout est ressource, chaque ressource a une addresse
  \item L'adresse est une URI
  \item Bien : http://somehost.com/zen/contacts/1
  \item Pas bien : http://somehost.com/zen/contacts?id=1
  \begin{itemize}
   \item L'identifiant fait partie de l'adresse
   \item Il doit être dans l'URL, pas en tant que paramètre
  \end{itemize}
 \end{itemize} 

\end{frame}

\begin{frame}
 \frametitle{Interface uniforme}
 
 \begin{itemize}
  \item Le client effectue des opérations sur une ressource
  \item Opérations disponibles :
  \begin{itemize}
   \item GET : récupération d'une ressource
   \item POST : créer une ressource
   \item PUT : modifier une ressource
   \item DELETE : supprimer une ressource
   \item HEAD : GET, mais sans le contenu, juste les entêtes
   \item OPTIONS : options de communication de la ressource
  \end{itemize}
 \end{itemize} 

\end{frame}

\begin{frame}
 \frametitle{Interface uniforme}
 
 \begin{itemize}
  \item Entêtes standardisés
  \begin{itemize}
   \item type de la requête, type attendu, taille de la réponse, etc.
  \end{itemize}
  \item Codes réponse standardisés
  \begin{itemize}
   \item 200 OK
   \item 201 Created
   \item 404 Not found
   \item 409 Conflict
   \item 500 Internal server error
   \item etc.
  \end{itemize}
 \end{itemize} 

\end{frame}

\begin{frame}
 \frametitle{Sans état}
 
 \begin{itemize}
  \item Aucun lien entre deux requêtes...
  \item Même si envoyées par le même client
  \item Facilite la distribution à grande échelle (``scalability'')
  \item Si état il y a, il est stocké dans une base de données
  \item Plus de session !
  \begin{itemize}
   \item En théorie...
   \item On peut choisir de ne pas suivre ce principe
  \end{itemize}
 \end{itemize} 

\end{frame}

\begin{frame}
 \frametitle{Représentation}
 
 \begin{itemize}
  \item Pas de format imposé pour représenter les ressources
  \item Formats courants :
  \begin{itemize}
   \item XML
   \item JSON
   \item ATOM
  \end{itemize}
  \item Possibilité d'utiliser des schémas
 \end{itemize} 

\end{frame}

\begin{frame}
 \frametitle{Hypermedia}
 
 \begin{itemize}
  \item Les ressources ont des liens vers d'autres ressources
  \item Exactement comme des pages web
  \item Un client peut suivre les liens d'une ressource à l'autre
  \begin{itemize}
   \item Un client intelligent...
  \end{itemize}
  \item Ex. : lien pour avancer dans un workflow ou l'annuler
  \item Généralement, entente entre client et fournisseur du service
 \end{itemize} 

\end{frame}