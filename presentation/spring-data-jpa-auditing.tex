\section{Auditing}

\begin{frame}
 \frametitle{Outline}
 \scriptsize{\tableofcontents[currentsection,currentsubsection]}
\end{frame}


\begin{frame}
 \frametitle{Problématique}
 \begin{itemize}
  \item Savoir qui a créé/modifié une entité
  \item Savoir quand l'entité a été créée/modifiée
  \item Spring Data JPA propose d'auditer les entités automatiquement
 \end{itemize}
\end{frame}

\begin{frame}[fragile]
 \frametitle{Déclarer le listener dans \code{/META-INF/orm.xml}}

 \begin{xmlcode}
<?xml version="1.0" encoding="UTF-8"?>
<entity-mappings>

  <persistence-unit-metadata>
    <persistence-unit-defaults>
      <entity-listeners>
        <entity-listener 
          class="o.s.data.jpa.domain.support.AuditingEntityListener" />
      </entity-listeners>
    </persistence-unit-defaults>
  </persistence-unit-metadata>

</entity-mappings>
 \end{xmlcode}
\end{frame}

\begin{frame}[fragile]
 \frametitle{Avoir une classe ``auditor''}

 \begin{itemize}
  \item Généralement le \code{User} de l'application
 \end{itemize}

 \begin{javacode}
@Entity
public class User {

  @Id
  @GeneratedValue(strategy = GenerationType.SEQUENCE)
  private Long id;

  private String login;
  
  (...)
}
 \end{javacode}
\end{frame}

\begin{frame}[fragile]
 \frametitle{Les entités à auditer doivent être \code{Auditable}}

 \begin{javacode}
public interface Auditable<U, ID extends Serializable> 
                 extends Persistable<ID> {

  U getCreatedBy();

  void setCreatedBy(final U createdBy);
  
  DateTime getCreatedDate();

  void setCreatedDate(final DateTime creationDate);

  U getLastModifiedBy();

  void setLastModifiedBy(final U lastModifiedBy);
  
  DateTime getLastModifiedDate();

  void setLastModifiedDate(final DateTime lastModifiedDate);
}
 \end{javacode}
\end{frame}

\begin{frame}[fragile]
 \frametitle{Rendre une entité \code{Auditable}}

 \begin{itemize}
  \item Soit ajouter les champs soi-même
 \end{itemize}

 \begin{javacode}
@Entity
public class Contact implements Auditable<User,Long> { 
  (...) 
  @ManyToOne
  private U createdBy;

  @Temporal(TemporalType.TIMESTAMP)
  private Date createdDate;

  @ManyToOne
  private U lastModifiedBy;

  @Temporal(TemporalType.TIMESTAMP)
  private Date lastModifiedDate;  
  (...)   
}
 \end{javacode}
\end{frame}

\begin{frame}[fragile]
 \frametitle{Rendre une entité \code{Auditable}}

 \begin{itemize}
  \item Soit utiliser une classe fournie par Spring Data JPA
 \end{itemize}

 \begin{javacode}
@Entity
public class Contact extends AbstractAuditable<User, Long> {
 
 (...)
  
}
 \end{javacode}
\end{frame}

\begin{frame}[fragile]
 \frametitle{Fournir l'utilisateur en cours}

 \begin{itemize}
  \item Implémenter \code{AuditorAware}
  \item Utilise généralement le contexte de sécurité
 \end{itemize}

 \begin{javacode}
public class SecurityAuditorAware implements AuditorAware<User> {

  @Override
  public User getCurrentAuditor() {
    return SecurityContext.getCurrentUser();
  }

}
 \end{javacode}
\end{frame}

\begin{frame}[fragile]
 \frametitle{Activer l'auditing}

 \begin{itemize}
  \item Déclarer le bean \code{AuditorAware}
  \item L'enregistrer auprès de Spring Data JPA
 \end{itemize}

 \begin{xmlcode}
<bean id="auditorAware" 
      class="com.zenika.nordnet.model.SecurityAuditorAware" />

<jpa:auditing auditor-aware-ref="auditorAware" />
 \end{xmlcode}
\end{frame}

\begin{frame}[fragile]
 \frametitle{Enjoy!}

 \begin{javacode}
Contact contact = new Contact();
contact.setFirstname("Joe");
contact.setLastname("Dalton");
// l'utilisateur doit \^etre positionn\'e sur l'AuditorAware !
contactRepository.save(contact);
// lastModifiedBy, lastModifiedDate, etc. positionn\'es !
 \end{javacode}
\end{frame}