\section{Tests d'intégration hors-conteneur}
\input{current-section.tex}

\begin{frame}
 \frametitle{Principes}
 \begin{itemize}
  \item Tester toute la stack Spring MVC
  \begin{itemize}
   \item validation, (dé)sérialisation, codes réponse, etc.
  \end{itemize}
  \item Sans utiliser un conteneur web
  \begin{itemize}
   \item Pas de requête HTTP
  \end{itemize}
 \end{itemize}
\end{frame}

\begin{frame}
 \frametitle{Pourquoi ?}
 
 \begin{itemize}
  \item Tester toute la couche web
  \begin{itemize}
   \item Cas de base et cas limites
  \end{itemize}
  \item Utiliser des mock pour les services métiers
  \begin{itemize}
   \item Plus facile pour simuler tous les cas
  \end{itemize}
 \end{itemize} 

\end{frame}

\begin{frame}[fragile]
 \frametitle{Configuration du test}
 
 \begin{itemize}
  \item Ajouter \code{@WebAppConfiguration}
  \item Injecter le contexte Spring
 \end{itemize}

 
 \begin{javacode}
@RunWith(SpringJUnit4ClassRunner.class)
@ContextConfiguration
@WebAppConfiguration
public class ContactControllerTest {

  @Autowired
  WebApplicationContext ctx;
  
  ...
}
 \end{javacode}

\end{frame}

\begin{frame}[fragile]
 \frametitle{Configuration dans le test}
 
 \begin{itemize}
  \item Classe interne statique, détectée automatiquement
 \end{itemize}

 
 \begin{javacode}
public class ContactControllerTest {

  ...
  
  @Configuration
  @EnableWebMvc
  @ComponentScan(basePackageClasses = ContactController.class)
  public static class TestConfiguration {

      @Bean
      public ContactRepository contactRepository() {
          return mock(ContactRepository.class);
      }

  }
}
 \end{javacode}

\end{frame}

\begin{frame}[fragile]
 \frametitle{Initialisation de \code{MockMvc}}
 
 \begin{javacode}
public class ContactControllerTest {

    @Autowired
    WebApplicationContext ctx; // le contexte Spring

    @Autowired ContactRepository repo; // une d\'ependance mock\'ee

    MockMvc mockMvc; // Spring MVC, version mock

    @Before public void setUp() {
        this.mockMvc = webAppContextSetup(ctx) // m\'ethode statique 
          .build();                            // Spring MVC test
        reset(repo); // m\'ethode statique Mockito
    }
    ...
}
 \end{javacode}


\end{frame}

\begin{frame}[fragile]
 \frametitle{Fluent API, le prix à payer}
 
 \begin{javacode}
import static o.m.Mockito.*;
import static o.s.test.web.servlet.request.MockMvcRequestBuilders.*;
import static o.s.test.web.servlet.result.MockMvcResultMatchers.*;
import static o.s.test.web.servlet.setup.MockMvcBuilders.*;
import static o.s.test.web.servlet.result.MockMvcResultHandlers.*;
 \end{javacode}

 
\end{frame}

\begin{frame}[fragile]
 \frametitle{Configuration du test}
 
 \begin{javacode*}{fontsize=\tiny}
@RunWith(SpringJUnit4ClassRunner.class)
@ContextConfiguration
@WebAppConfiguration
public class ContactControllerTest {

    @Autowired WebApplicationContext ctx;

    @Autowired ContactRepository repo;

    MockMvc mockMvc;

    @Before public void setUp() {
        this.mockMvc = webAppContextSetup(ctx).build();
        reset(repo);
    }
    ...
    @Configuration
    @EnableWebMvc
    @ComponentScan(basePackageClasses = ContactController.class)
    public static class TestConfiguration {

        @Bean
        public ContactRepository contactRepository() {
            return mock(ContactRepository.class);
        }

    }
}   
 \end{javacode*}
 
\end{frame}


\begin{frame}[fragile]
 \frametitle{Méthode à tester}
 
 \begin{javacode}
@Controller
public class ContactController {

  @Autowired ContactRepository contactRepository;

  @RequestMapping(value="/contacts/{id}",method=RequestMethod.GET)
  @ResponseBody
  public Contact contact(@PathVariable Long id) {		
    Contact contact = contactRepository.findOne(id);
    if(contact == null) {
      throw new EmptyResultDataAccessException(1);
    }
    return contact;
  }
  
}
 \end{javacode}

 
\end{frame}

\begin{frame}[fragile]
 \frametitle{Lancer une requête}
 
 \begin{javacode}
@Test
public void contactExists() throws Exception {
    Long id = 1L;
    when(repo.findOne(id)).thenReturn(new Contact(id,"John","Doe",33));
    mockMvc.perform(get("/contacts/{id}", id))
         .andDo(print()); // affichage requ\^ete/r\'eponse dans la console
}
 \end{javacode}

\end{frame}

\begin{frame}[fragile]
 \frametitle{Voir ce qui se passe}
 
 \begin{javacode}
mockMvc.perform(get("/contacts/{id}", id)).andDo(print());
 \end{javacode}
 
 \begin{textcode}
...
MockHttpServletResponse:
          Status = 200
  Error message = null
        Headers = {Content-Type=[application/json;charset=UTF-8]}
   Content type = application/json;charset=UTF-8
           Body = {"id":1,"firstname":"John","lastname":"Doe","age":33}
  Forwarded URL = null
 Redirected URL = null
        Cookies = []
...
 \end{textcode}


\end{frame}

\begin{frame}[fragile]
 \frametitle{Tester un GET}
 
 \begin{javacode}
@Test
public void contactExists() throws Exception {
    Long id = 1L;
    when(repo.findOne(id)).thenReturn(new Contact(id,"John","Doe",33));
    mockMvc.perform(get("/contacts/{id}", id))
         .andExpect(status().isOk())
         .andExpect(jsonPath("id").value(1))
         .andExpect(jsonPath("firstname").value("John"))
         .andExpect(jsonPath("lastname").value("Doe"))
         .andExpect(jsonPath("age").value(33));
}
 \end{javacode}

\end{frame}

\begin{frame}[fragile]
 \frametitle{Si la ressource n'existe pas...}
 
 \begin{javacode}
@Controller
public class ContactController {

  @Autowired ContactRepository contactRepository;

  @RequestMapping(value="/contacts/{id}",method=RequestMethod.GET)
  @ResponseBody
  public Contact contact(@PathVariable Long id) {		
    Contact contact = contactRepository.findOne(id);
    if(contact == null) {
      throw new EmptyResultDataAccessException(1);
    }
    return contact;
  }
  
  @ExceptionHandler(EmptyResultDataAccessException.class)
  @ResponseStatus(HttpStatus.NOT_FOUND)
  public void notFound() { }
  
}
 \end{javacode}

 
\end{frame}

\begin{frame}[fragile]
 \frametitle{Tester un GET qui retourne 404}
 
 \begin{javacode}
@Test
public void contactDoesNotExists() throws Exception {
    Long id = 1L;
    when(repo.findOne(1L)).thenReturn(null);
    mockMvc.perform(get("/contacts/{id}", id))
            .andExpect(status().isNotFound());
}
 \end{javacode}

\end{frame}



\begin{frame}[fragile]
 \frametitle{Un GET qui retourne un tableau}
 
 \begin{textcode}
[ 
  {"id":1,"firstname":"John","lastname":"Doe","age":33},
  {"id":2,"firstname":"Jane","lastname":"Doe","age":30}
]
 \end{textcode} 

 \begin{javacode}
@Test public void contacts() throws Exception {
    when(repo.findAll()).thenReturn(Arrays.asList(
            new Contact(1L,"John","Doe",33),
            new Contact(2L,"Jane","Doe",30)
    ));
    mockMvc.perform(get("/contacts"))
        .andExpect(status().isOk())
        .andExpect(jsonPath("$[0].id").value(1))
        .andExpect(jsonPath("$[0].firstname").value("John"))
        .andExpect(jsonPath("$[0].lastname").value("Doe"))
        .andExpect(jsonPath("$[1].id").value(2))
        .andExpect(jsonPath("$[1].firstname").value("Jane"))
        .andExpect(jsonPath("$[1].lastname").value("Doe"));
}
 \end{javacode}

\end{frame}

\begin{frame}[fragile]
 \frametitle{Une méthode POST avec un corps}
 
 \begin{javacode}
@RequestMapping(value="/contacts",method=RequestMethod.POST)
@ResponseStatus(HttpStatus.CREATED)
public void create(@RequestBody Contact contact, 
                   HttpServletRequest request, 
                   HttpServletResponse response) {
  contact = contactRepository.save(contact);
  String location = ServletUriComponentsBuilder.fromRequest(request)
    .pathSegment("{id}")
    .build()
    .expand(contact.getId())
    .encode()
    .toUriString();
  response.setHeader("Location", location);
}
 \end{javacode}

\end{frame}

\begin{frame}[fragile]
 \frametitle{Tester une méthode POST avec un corps}
 
 \begin{javacode}
@Test public void create() throws Exception {
    Contact toBeCreated = new Contact(1L,"John","Doe",33);

    when(repo.save(any(Contact.class))).thenReturn(toBeCreated);
    mockMvc.perform(post("/contacts")
        .content(
          "{\"firstname\":\"John\",\"lastname\":\"Doe\",\"age\":33}"
        )
        .contentType(MediaType.APPLICATION_JSON))
      .andExpect(status().isCreated())
      .andExpect(header().string(
         "Location","http://localhost/contacts/1")
      );

    (...)
}
 \end{javacode}

\end{frame}

\begin{frame}[fragile]
 \frametitle{Tester une méthode POST avec un corps (suite)}
 
 \begin{javacode}
@Test public void create() throws Exception {
    Contact toBeCreated = new Contact(1L,"John","Doe",33);

    (...)
    // JSON => Contact OK ?
    ArgumentCaptor<Contact> contactCaptor = ArgumentCaptor.forClass(
      Contact.class
    );
    verify(repo).save(contactCaptor.capture());
    Contact captured = contactCaptor.getValue();
    Assert.assertEquals(toBeCreated.getFirstname(),
      captured.getFirstname());
    Assert.assertEquals(toBeCreated.getLastname(),
      captured.getLastname());
    Assert.assertEquals(toBeCreated.getAge(),
      captured.getAge());
}
 \end{javacode}

\end{frame}