\section{REST}
\begin{frame}
 \frametitle{Outline}
 \scriptsize{\tableofcontents[currentsection,currentsubsection]}
\end{frame}


\begin{frame}
 \frametitle{Spring MVC et REST}
 \begin{itemize}
  \item Spring MVC fournit un support REST
  \item Même modèle de programmation que pour les applications web
  \item Annotations et mécanismes supplémentaires
  \item Principale différence : plus de vue
  \item Spring MVC
  \begin{itemize}
   \item désérialise le contenu des requêtes
   \item sérialise le contenu des réponses
  \end{itemize}
 \end{itemize}
\end{frame}

\begin{frame}[fragile]
 \frametitle{Activer le support REST}

 \begin{itemize}
  \item Utiliser le starter web de Spring Boot
  \item Inclut Spring MVC, sérialisation JSON, etc
 \end{itemize}

 \begin{xmlcode}
<dependency>
  <groupId>org.springframework.boot</groupId>
  <artifactId>spring-boot-starter-web</artifactId>
</dependency>
 \end{xmlcode}

\end{frame}

\begin{frame}[fragile]
 \frametitle{Récupérer une ressource (HTTP)}

 \begin{itemize}
  \item Requête
 \end{itemize}

 \begin{textcode}
GET /contacts/1 HTTP/1.1
Accept: application/json
Host: localhost:8080
 \end{textcode}

  \begin{itemize}
  \item Réponse
 \end{itemize}

 \begin{textcode}
HTTP/1.1 200 OK
Content-Type: application/json
{"id":1,"firstname":"Joe","lastname":"Dalton","age":37}
 \end{textcode}

\end{frame}

\begin{frame}[fragile]
 \frametitle{Récupérer une ressource (contrôleur)}

 \begin{javacode}
@RequestMapping(value="/contacts/{id}",method=RequestMethod.GET)
public Contact contact(@PathVariable Long id) {
  return contactRepository.findOne(id);
}
 \end{javacode}

\end{frame}

\begin{frame}[fragile]
 \frametitle{Récupérer une ressource (contrôleur)}

 \begin{javacode}
@RequestMapping(
  value="/contacts/{id}",     // URL
  method=RequestMethod.GET    // op\'eration
)
public Contact contact(
    @PathVariable Long id) {  // \`a extraire de l'URL
  return contactRepository.findOne(id);
}
 \end{javacode}

\end{frame}

\begin{frame}[fragile]
 \frametitle{Récupérer une ressource (client)}

 \begin{javacode}
RestTemplate tpl = new RestTemplate();
Contact contact = tpl.getForObject(
  "http://localhost:8080/contacts/{id}",
  Contact.class,
  id
);
 \end{javacode}

\end{frame}

\begin{frame}[fragile]
 \frametitle{Récupérer des ressources (HTTP)}

 \begin{itemize}
  \item Requête
 \end{itemize}

 \begin{textcode}
GET /contacts HTTP/1.1
Accept: application/json
Host: localhost:8080
 \end{textcode}

  \begin{itemize}
  \item Réponse
 \end{itemize}

 \begin{textcode}
HTTP/1.1 200 OK
Content-Type: application/json
[
 {"id":1,"firstname":"Joe","lastname":"Dalton","age":37},
 {"id":2,"firstname":"William","lastname":"Dalton","age":35},
 {"id":3,"firstname":"Jack","lastname":"Dalton","age":33},
 {"id":4,"firstname":"Averell","lastname":"Dalton","age":31}
]
 \end{textcode}

\end{frame}

\begin{frame}[fragile]
 \frametitle{Récupérer des ressources (contrôleur)}

 \begin{javacode}
@RequestMapping(value="/contacts",method=RequestMethod.GET)
public List<Contact> contacts() {
  return contactRepository.findAll();
}
 \end{javacode}

\end{frame}

\begin{frame}[fragile]
 \frametitle{Récupérer des ressources (client)}

 \begin{javacode}
RestTemplate tpl = new RestTemplate();
Contact [] contacts = tpl.getForObject(
  "http://localhost:8080/contacts",
  Contact[].class
);
 \end{javacode}

\end{frame}

\begin{frame}[fragile]
 \frametitle{Créer une ressource (HTTP)}

 \begin{itemize}
  \item Requête
 \end{itemize}

 \begin{textcode}
POST /contacts HTTP/1.1
Content-Type: application/json;charset=UTF-8
Host: localhost:8080
{"id":null,"firstname":"Oncle","lastname":"Picsou","age":100}
 \end{textcode}

  \begin{itemize}
  \item Réponse
 \end{itemize}

 \begin{textcode}
HTTP/1.1 201 Created
Location: http://localhost:8080/contacts/130
Content-Length: 0
 \end{textcode}

\end{frame}

\begin{frame}[fragile]
 \frametitle{Créer une ressource (contrôleur)}

 \begin{javacode}
@RequestMapping(value="/contacts",method=RequestMethod.POST)
public ResponseEntity<Void> create(@RequestBody Contact contact,
                          UriComponentsBuilder uriComponentsBuilder) {
  contactRepository.save(contact);
 	URI location = uriComponentsBuilder
    .pathSegment("contacts","{id}")
    .build()
    .expand(contact.getId())
    .encode()
    .toUri();
 	return ResponseEntity.created(location).build();
}
 \end{javacode}

\end{frame}

\begin{frame}[fragile]
 \frametitle{Créer une ressource (contrôleur)}

 \begin{javacode}
@RequestMapping(value="/contacts",method=RequestMethod.POST)
public ResponseEntity<Void> create(
      @RequestBody Contact contact,  // extraire du corps de requ\^ete
      UriComponentsBuilder uriComponentsBuilder) {
  (...)
}
 \end{javacode}

\end{frame}

\begin{frame}[fragile]
 \frametitle{Créer une ressource (contrôleur)}

 \begin{javacode}
@RequestMapping(value="/contacts",method=RequestMethod.POST)
public ResponseEntity<Void> create(@RequestBody Contact contact,
                          UriComponentsBuilder uriComponentsBuilder) {
  (...)
 	return ResponseEntity.created(     // code r\'eponse
    location                         // URL de la ressource cr\'e\'ee
  ).build();
}
 \end{javacode}

\end{frame}

\begin{frame}[fragile]
 \frametitle{Créer une ressource (client)}

 \begin{javacode}
Contact contact = new Contact();
contact.setFirstname("Oncle");
contact.setLastname("Picsou");
contact.setAge(100);
RestTemplate tpl = new RestTemplate();
URI location = tpl.postForLocation(
  "http://localhost:8080/contacts",
  contact
);
 \end{javacode}

\end{frame}

\begin{frame}[fragile]
 \frametitle{Modifier une ressource (HTTP)}

 \begin{itemize}
  \item Requête
 \end{itemize}

 \begin{textcode}
PUT /contacts/130 HTTP/1.1
Content-Type: application/json;charset=UTF-8
Host: localhost:8080
{"id":130,"firstname":"Oncle","lastname":"Picsou","age":90}
 \end{textcode}

  \begin{itemize}
  \item Réponse
 \end{itemize}

 \begin{textcode}
HTTP/1.1 204 Not Content
Content-Length: 0
 \end{textcode}

\end{frame}

\begin{frame}[fragile]
 \frametitle{Modifier une ressource (contrôleur)}

 \begin{javacode}
@RequestMapping(value="/contacts/{id}",method=RequestMethod.PUT)
@ResponseStatus(HttpStatus.NO_CONTENT)  // code r\'eponse
public void update(@RequestBody Contact contact) {
  contactRepository.save(contact);
}
 \end{javacode}

\end{frame}

\begin{frame}[fragile]
 \frametitle{Modifier une ressource (client)}

 \begin{javacode}
RestTemplate tpl = new RestTemplate();
// cr\'eation donne l'URI
URI location = tpl.postForLocation(...);
// modification
contact.setAge(90);
tpl.put(location,contact);
 \end{javacode}

\end{frame}

\begin{frame}[fragile]
 \frametitle{Supprimer une ressource (HTTP)}

 \begin{itemize}
  \item Requête
 \end{itemize}

 \begin{textcode}
DELETE /contacts/130 HTTP/1.1
Host: localhost:8080
 \end{textcode}

  \begin{itemize}
  \item Réponse
 \end{itemize}

 \begin{textcode}
HTTP/1.1 204 Not Content
Content-Length: 0
 \end{textcode}

\end{frame}

\begin{frame}[fragile]
 \frametitle{Supprimer une ressource (contrôleur)}

 \begin{javacode}
@RequestMapping(value="/contacts/{id}",method=RequestMethod.DELETE)
@ResponseStatus(HttpStatus.NO_CONTENT)
public void delete(@PathVariable Long id) {
  contactRepository.delete(id);
}
 \end{javacode}

\end{frame}

\begin{frame}[fragile]
 \frametitle{Supprimer une ressource (client)}

 \begin{javacode}
RestTemplate tpl = new RestTemplate();
// cr\'eation donne l'URI
URI location = tpl.postForLocation(...);
// suppression
tpl.delete(location);
 \end{javacode}

\end{frame}

\begin{frame}
 \frametitle{Gestion des erreurs}

 \begin{itemize}
  \item Ex. : demande d'une ressource qui n'existe pas
  \item Le serveur doit retourner une erreur 404
  \item Comment faire avec Spring MVC, cotés serveur et client ?
 \end{itemize}

\end{frame}

\begin{frame}[fragile]
 \frametitle{Gestion des erreurs, coté serveur, solution 1}

 \begin{javacode}
@RequestMapping(value="/contacts/{id}",method=RequestMethod.GET)
public Contact contact(@PathVariable Long id) {
  Contact contact = contactRepository.findOne(id);
  if(contact == null) {
    throw new EmptyResultDataAccessException(1);
  }
  return contact;
}

@ExceptionHandler(EmptyResultDataAccessException.class)
@ResponseStatus(HttpStatus.NOT_FOUND)
public void notFound() { }
 \end{javacode}

\end{frame}

\begin{frame}[fragile]
 \frametitle{Gestion des erreurs, coté serveur, solution 2}

 \begin{itemize}
  \item Retourner une \code{@ResponseEntity}
 \end{itemize}

 \begin{javacode}
@RequestMapping(value="/contacts/{id}",method=RequestMethod.GET)
public ResponseEntity<Contact> contact(@PathVariable Long id) {
  Contact contact = contactRepository.findOne(id);
  ResponseEntity<Contact> response = new ResponseEntity<Contact>(
    contact,
    contact == null ? HttpStatus.NOT_FOUND : HttpStatus.OK
  );
  return response;
}
 \end{javacode}

\end{frame}

\begin{frame}[fragile]
 \frametitle{Gestion des erreurs, coté client}

 \begin{javacode}
try {
  Contact contact = tpl.getForObject(location, Contact.class);
} catch (HttpStatusCodeException e) {
  // e.getStatusCode() == HttpStatus.NOT\_FOUND
}
 \end{javacode}

 \begin{itemize}
  \item Stratégie de gestion des exceptions configurable
  \item Propriété \code{errorHandler} du \code{RestTemplate}
 \end{itemize}

\end{frame}
