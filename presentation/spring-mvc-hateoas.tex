\section{HATEOAS}
\input{current-section.tex}

\begin{frame}
 \frametitle{HATEOAS}
 \begin{itemize}
  \item Hypermedia as the engine of application state
  \item Permet de découvrir les actions futures/possibles
  \begin{itemize}
   \item Naviguer vers une autre ressource
   \item Avoir le détail d'une ressource
   \item Récupérer d'autres représentations d'une ressource
  \end{itemize}
 \end{itemize}
\end{frame}

\begin{frame}[fragile]
 \frametitle{HATEOAS : exemple}
 \begin{textcode}
[
  {
    "id":{
      "rel":"self",
      "href":"http://localhost:8080/hateoas/zen-contact/contacts/1"
    },
    "firstname":"Joe",
    "lastname":"Dalton",
    "links":[
      {
        "rel":"self",
        "href":"http://localhost:8080/hateoas/zen-contact/contacts/1"
      }
    ]
  }, ...
]
 \end{textcode}

\end{frame}

\begin{frame}
 \frametitle{Spring HATEOAS}
 \begin{itemize}
  \item Une librairie proposant un support HATEOAS
  \item S'intégre avec Spring MVC
  \item En cours de développement !
 \end{itemize}

\end{frame}

\begin{frame}[fragile]
 \frametitle{\code{Link}}
 
 \begin{javacode}
Link link = new Link(
  "http://localhost:8080/hateoas/zen-contact/contacts/1",
  Link.REL_SELF
);
 \end{javacode}

 \begin{itemize}
  \item \code{Link}s ajoutés aux ressources...
  \item ... puis sérialisés en JSON, XML...
 \end{itemize}

\end{frame}

\begin{frame}[fragile]
 \frametitle{Ressource avec des liens}
 
 \begin{itemize}
  \item Rajouter un support pour \code{Link} dans ses ressources...
  \item ... ou utiliser \code{ResourceSupport}
 \end{itemize}
 
 \begin{javacode}
public class ShortContact extends ResourceSupport {

  private String firstname,lastname;
  (...) // getters and setters
}
...
ShortContact resource = new ShortContact();
resource.add(new Link("http://localhost/contacts/1"));
 \end{javacode}

\end{frame}

\begin{frame}[fragile]
 \frametitle{Intégration avec Spring MVC}

 \begin{javacode}
import static org.springframework.hateoas.mvc.ControllerLinkBuilder.*;

@Controller
@RequestMapping("/contacts")
public class ContactController { 
  ...
  
  // dans une m\'ethode
  // lien \`a partir de l'URL du contr\^oleur, ajout d'un identifiant
  Link detail = linkTo(ContactController.class)
    .slash(contact.getId())
    .withSelfRel();

}
 \end{javacode}

\end{frame} 