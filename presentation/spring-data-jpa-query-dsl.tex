\section{Querydsl}

\begin{frame}
 \frametitle{Outline}
 \scriptsize{\tableofcontents[currentsection,currentsubsection]}
\end{frame}


\begin{frame}
 \frametitle{Querydsl}
 \begin{itemize}
  \item API pour la construction de requêtes
  \item Type-safe si le méta-modèle est généré
  \begin{itemize}
   \item C'est le cas dans les exemples suivants
  \end{itemize}
  \item Support pour JPA, Hibernate, JDO, SQL, MongoDB, etc.
  \item Projet Open Source (\url{http://www.querydsl.com/})
 \end{itemize}
\end{frame}

\begin{frame}[fragile]
 \frametitle{Une requête Querydsl sur \code{Contact}}

 \begin{javacode}
JPAQuery query = new JPAQuery(entityManager);
QContact contact = QContact.contact; // m\'eta-mod\`ele

List<Contact> = query.from(contact)
  .where(
    contact.lastname.eq("Dalton").and(contact.age.between(30, 40))
  )
  .list(contact);		
 \end{javacode}
\end{frame}

\begin{frame}
 \frametitle{Querydsl vs. Criteria JPA 2.0}

 \begin{itemize}
  \item Tous les deux type-safe grâce à la génération du méta-modèle
  \item Querydsl a fait le choix d'une API courante (``fluent API'')
  \item Tous deux nécessitent un temps d'apprentissage
 \end{itemize}

\end{frame}

\begin{frame}[fragile]
 \frametitle{\code{Predicate}}

 \begin{itemize}
  \item Querydsl définit une interface \code{Predicate}
  \begin{itemize}
   \item Ne pas confondre avec celle de JPA 2.0
  \end{itemize}
  \item \code{BooleanExpression} est l'implémentation la plus courante
 \end{itemize}

 \begin{javacode}
BooleanExpression predicate = QContact.contact.lastname.eq("Dalton");
 \end{javacode}
\end{frame}

\begin{frame}[fragile]
 \frametitle{Spring Data JPA et Querydsl}

 \begin{javacode}
package org.springframework.data.querydsl;

public interface QueryDslPredicateExecutor<T> {

  T findOne(Predicate predicate);

  Iterable<T> findAll(Predicate predicate);

  Iterable<T> findAll(Predicate predicate, OrderSpecifier<?>...orders);

  Page<T> findAll(Predicate predicate, Pageable pageable);

  long count(Predicate predicate);
}
 \end{javacode}
\end{frame}

\begin{frame}[fragile]
 \frametitle{Rendre le repository \code{Predicate}-friendly}

 \begin{itemize}
  \item Etendre l'interface \code{QueryDslPredicateExecutor<T>}
 \end{itemize}

 \begin{javacode}
public interface ContactRepository 
	    extends Repository<Contact,Long>,
	    QueryDslPredicateExecutor<Contact> {

}
 \end{javacode}

\end{frame}

\begin{frame}[fragile]
 \frametitle{Pattern spécification pour les \code{Predicate}s}

 \begin{itemize}
  \item Pour la réutilisation/combinaison des \code{Predicate}s
  \item Moins justifié qu'avec JPA 2.0
  \begin{itemize}
   \item \code{Predicate}s plus simples à construire
  \end{itemize}
 \end{itemize}

\begin{javacode}
public abstract class ContactSpecs {

  public static BooleanExpression outlaws() {
    return QContact.contact.lastname.eq("Dalton");
  }

  public static BooleanExpression inMidThirties() { (...) }

}
\end{javacode}

\end{frame}

\begin{frame}[fragile]
 \frametitle{Utiliser les \code{Predicate}s}

 \begin{javacode}
import static com.zenika.nordnet.repository.ContactSpecs.*;
(...)
// 1 Predicate
int count = repo.count(outlaws());
// Combinaison de Predicates (utilisation de BooleanExpression)
int count = repo.count(outlaws().and(inMidThirties()));
 \end{javacode}

\end{frame}